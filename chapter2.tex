% !TEX encoding = UTF-8
%毛概
\section{毛概部分}
\textcolor{red}{\faBook 毛概的部分很多也很难,需要记忆的东西最多,习是真的能写。}
\begin{enumerate}[align=hang, start=1]
	%第一题
	\item 1938年,毛泽东在党的六届六中全会上作的题为《论新阶段》的政治报告中,最先提出了``马克思主义中国化''的命题。毛泽东提出要实现马克思主义的中国化,源于
	\xx{对马克思主义的深刻理解}{对中国国情的准确把握}{\textcolor{red}{对中国革命进程中正反两个反面的实践经验的科学总结}}{对当时时代主题的清晰判断}
	\note B、C选项容易纠结不清楚,三短一长选最长吧。
	
	%第二题
	\item 下列各项中,体现毛泽东思想在解放战争时期和新中国成立以后继续得到发展的有
	\xx{\textcolor{red}{提出了人民民主专政的理论}}{\textcolor{red}{提出了社会主义改造和建设社会主义制度的基本方略 }}{\textcolor{red}{ 提出了要把马克思主义的基本理论同中国革命和建设具体相结合的``第二次结合''}}{\textcolor{red}{ 提出了关于正确处理人民内部矛盾的理论}}
	\note 漏选了C项,精练精讲有明确的回答,P105。
	
	%第三题
	\item 毛泽东提出,中国共产党区别于其他任何政党的显著标志是 
	\xx {  \textcolor{red}{理论和实践相结合的作风}  } { 
		\textcolor{red}{和人民群众紧密联系在一起的作风} }    { \textcolor{red}{自我批评的作风} }   { 谦虚、谨慎、不骄、不躁的作风 }
	\note 而区别于任何政党的标志是毛泽东在党的七大作的 \trq{《论联合政府》}中指出的。区别于任何政党的标志是A、B、C三项,而D是``两个务必''的内容,\textcolor{teal}{即``务必使同志们继续保持谦虚、谨慎、不骄、不躁''的作风},\textcolor{teal}{务必使同志们继续地保持艰苦奋斗的作风},P121。
	
	%第四题
	\item 实事求是,就是一切从实际出发,理论联系实际,不断深化对中国国情的认识,找出适合中国情况的革命和建设道路,确定我们党领导人民改造中国、建设中国的战略策略,实现推动历史前进的目标。实事求是,是
	\xx {   	\textcolor{red}{ 马克思主义中国化理论成果的精髓和灵魂 }}    { 	\textcolor{red}{ 中国共产党人认识世界、改造世界的根本要求}  }    {	\textcolor{red}{ 中国共产党的基本思想方法、工作方法、领导方法 }   }  { 	\textcolor{red}{    马克思主义的根本观点}    }
	\note  漏选了D,记住它的面容。  
	
	%第五题
	\item 近代以来中华民族面临的两大历史任务是 
	\xx{ 推翻军阀官僚的反动统治  } { 	\textcolor{red}{ 争取民族独立和人民解放 }}  {建立无产阶级专政的政权}  { 	\textcolor{red}{ 实现国家富强和人民富裕 }}
	\note 错选了A,近代以来这个范围还是很广的。
	
	%第六题
	\item 中国革命的中心问题是
	\xx { 农民土地问题}{统一战线问题}{ 武装斗争问题}{\tr{领导权问题}}
	\note 民主革命的几个提法,  中国革命的\tr{中心问题}、新民主主义革命理论的\tr{核心问题}是\tr{无产阶级领导权},
	中国革命的\tr{基本问题}是\tr{农民问题},新民主主义革命的\tr{主要内容},\tb{没收封建地主阶级的土地归农民所有},
	新民主主义革命的\tr{实质}是,\tb{中国共产党领导下的农民革命},新民主主义革命的\tr{首要问题是}\tb{分清敌友}。
	
	\item 
	工人阶级实现革命领导权的基础是
	\xx{\tr{对农民阶级的领导}}{对于小资产阶级的领导}{对于民族资产阶级的领导}{对于军队的领导} 
	\note 记住她
	
	%第八题
	\item  ``因为中国资产阶级根本上与剥削农民的豪绅地主相联合相混合,中国革命要推翻豪绅地主阶级,便不能不同时推翻资产阶级。''这一观点属于
	\xx{ \tr{``毕其功于一役''的观点}}{``二次革命''的观点}{民主主义革命是社会主义革命必要准备的观点}{中国革命分``两步走''的观点}
	\note ``毕其功于一役'',\tb{混淆了新民主主义革命和社会主义革命的界限},是一种\tr{``左倾主义''}的观点,``两步走'',\tb{割裂了新民主主义革命和社会主义革命的联系},是一种\tr{``右倾主义''}的观点。
	
	%第9题
	\item 
	新民主主义革命理论的核心问题是
	\xx{分清敌友}{农民问题}{\tr{无产阶级领导权}}{土地问题}
	\note \tq{新民主主义革命理论的核心问题和中国革命的中心问题都是}\tr{无产阶级领导权的问题}。	
	
	%第10题
	\item 
	新民主主义革命理论形成的客观条件有
	\xx{\tr{旧民主主义革命的失败} }{ \tr{近代中国革命形式的发展} }{  对中国革命经验教训的概括和总结 }{ \tr{世界形式的新变化} }
	\note C项属于理论的东西不属于客观的条件。
	
	%第11题
	\item 
	无产阶级及其政党对中国革命的领导权不是自然而然得来的,而是在与资产阶级争夺领导权的斗争中实现的。毛泽东提出的无产阶级要实现对同盟者的领导必备的条件有。
	\xx{ \tr{率领同盟者向共同的敌人做坚决的斗争,并取得胜利}}{坚持发展进步势力,争取中间势力,孤立顽固势力的方针}{\tr{对同盟者以物质福利,至少不损害其利益,同时给予政治教育}}{对同盟者采取``有理、有利、有节''的策略原则}
	\note A、C项是毛泽东提出的``领导的阶级和政党,要实现自己对于被领导阶级、阶层、政党和人民团体的领导,具备的条件'',B是抗日战争中共巩固和扩大民族统一战线的总方针,D是抗日战争中对顽固派势力的总方针。
	
	%第12题
	\item 
	新民主主义经济纲领的内容是
	\xx{\tr{没收封建地主阶级的土地归农民所有}}{ \tr{没收官僚资产阶级的垄断资本归新民主主义的国家所有}}{\tr{保护民族工商业}}{没收外国在华资本归新民主主义的国家所有}
	\note 记住她就行
	
	%第13题
	\item 
	关于新民主主义政治纲领规定的新民主主义国家,以下说法正确的有
	\xx { \tr{国体是各革命阶级的联合专政}} {\tr{政体是民主集中制的人民代表大会制度}}{各革命阶级的联合专政是工农民主专政}{\tr{人民当家作主是新民主主义国家制度的核心内容和基本准则}}
	\note \tq{国体:}\tb{无产阶级领导的以工农联盟为基础,包括小资产阶级、民族资产阶级和其他反帝反封建的人们在内的各革命阶级的联合专政},\tq{政体:}\tb{民主集中制的人民代表大会制度}。
	
	%第14题
	\item 
	新民主主义文化就是无产阶级领导的人民大众的反帝反封建的文化,即民族的科学的大众的文化。其中``民族的'',就其内容说,是指 
	\xx{具有鲜明的民族风格、民族形式和民族特色}{具有中国作风和中国气派}{\trq{反对帝国主义压迫,主张中华民族的尊严和独立}}{ 反对封建思想和迷信思想}
	\note \tb{新民主主义文化是\trq{民族的},就其\trq{内容}说是反对帝国主义压迫,主张中华民族的尊严和独立,就其\trq{形式}是具有鲜明的民族风格、民族形式和民族特色,要有中国作风和中国气派。}\tq{新民主主义文化中居于指导地位的是共产主义思想,而非新民主主义思想。}
	
	%第15题
	\item 中国革命的特点和优点是
	\xx{由中国共产党领导的人民战争}{目标是争取民族独立、人民解放,最终实现国家的繁荣富强}{以反帝反封建作为两大革命任务}{\trq{以武装的革命反对武装的反革命}} 
	\note \tb{武装斗争是中国革命的特点和优点之一。}
	``三个'为一组的易混淆的考点 
	
	%表格总结
	\begin{center}
			\hspace{-1cm}
		\begin{tabular}{|c|c|c|}
			 \hline 
			 提出时间和著作      &        名称            &        内容    \\ 
			 \hline 
			 \makecell[c]{1981年十一届六中全会\\《历史决议》 }         &     毛泽东思想活的灵魂      & 独立自主、实事求是、群众路线 \\
			 \hline
			 																		&  \makecell[c]{中国革命走农村包围城市,\\ 武装夺取政权的道路的三个内容} &     \makecell[c]{土地革命、武装斗争\\农村革命根据地建设}		\\
			\hline 
			\makecell[c]{1939年\\《共产党人发刊词》}		& 新民主主义革命的三大法宝     &    \makecell[c]{统一战线、武装斗争\\ 党的建设}	\\			    
			\hline
			\makecell[c]{1935年七大  \\ {  《论联合政府》}}    &中国共产党的三大优良作风   &   \makecell[c]{理论联系实际、密切联系群众 \\批评与自我批评}			\\
			\hline
		\end{tabular}
	\end{center}
	
	
	%第16题
	\item 
	土地革命战争时期,毛泽东指出:``一国之内,在四周白色政权的包围中,有一块或若干小块红色政权的区域长期存在,这是世界各国从来没有的事。这种奇事的发生,有其独特的原因''。红色政权能够存在和发展的原因是
	\xx{ \trq{近代中国是多个帝国主义间接统治的半殖民地国家,社会政治经济发展极端不平衡}}{\trq{过敏革命的政治影响及良好的群众基础}}{\trq{全国革命形式的继续向前发展}}{\trq{相当力量正式红军的存在以及当的有力量及其正确的政策}}
	\note 必须区分好\trq{``必须走''}和\trq{``可以走''},必要性是由\trq{具体国情}决定的,\tb{内无民主制度,外无民族独立}
	以及\tb{中国是一个农业大国,农民占全国人口的绝大多数,是无产阶级可靠的同盟军和革命的\trq{主力军}}。
	而``可以走''是由\trq{特殊国情}决定的,即为题目选项,其中 ``半殖半封''提供了缝隙和可能、革命群众基础好、全国革命形式的继续发展提供了\trq{客观条件}、相当力量的红军则是\trq{坚实的后盾},党的正确政策则是重要的\trq{主观条件}。必须走的两条还分别回答了为什么必须 \trq{武装夺取政权}和\trq{农村包围城市}的问题。
	
	%第17题
	\item 
	中国革命必须建立最广泛的统一战线,这是由
	\xx{\trq{中国半殖民地和半封建社会的阶级状况决定的}}{半殖民地半封建地的中国社会的交织在一起的诸多矛盾决定的}{\trq{中国革命的长期性、残酷性及发展的不平衡性所决定的}}{战争与革命的时代主题决定的}
	\note  这是建立革命统一战线的\trq{必要性},而B选项则是可能性。
	
	%第18题
	\item 
	近代中国社会的阶级结构是``两头小中间大'',``中间是指'' 
	\xx{\trq{无产阶级}}{农民阶级}{\trq{民族资产阶级}}{城市小资产阶级}
	\note \tq``两头''一头是指 \trq{无产阶级},另一头指 \trq{地主大资产阶级},``中间''是指 \trq{农民、城市小资产阶级以及其他中间势力}。
	
	
	
\end{enumerate}