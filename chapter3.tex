% !TEX encoding = UTF-8
%近代史部分
\section{近代史}

\textcolor{red}{\faBook 近代史的部分相对比较简单,主要是土地、还有一些革命性质弄清楚,串联成线}


\begin{enumerate}[align=hang, start=1]
	
	%第一题
	\item 鸦片战争后中国\textcolor{red}{社会性质}发生的变化是	
	\xx{\textcolor{red}{ 独立的中国逐步成为半殖半封的中国 }} { 帝国主义和中华民族的矛盾成为主要矛盾 }  { \textcolor{red}{ 封建的中国逐步成为半封建地的中国 }  } {  反对帝国主义成为中国革命的主要任务 } 
	\note  主义是性质的变化,B、D两项都是矛盾的变化。
	
	%第二题
	\item 下列条约中涉及香港的有 
	\xx { \textcolor{red}{ 《南京条约》}}{ 《辛丑条约》 } { \textcolor{red}{ 《北京条约》}} {\textcolor{red}{《展拓香港界址条约》}}
 	\note 1842年南京条约,香港岛割让,1860年北京条约,割去香港岛对岸九龙半岛南端和昂船洲,1898年展拓香港界址条约被迫签署。
 	
	%第三题
 	\item 在近代,帝国主义列强不能共同瓜分中国和对中国实行直接殖民统治的原因在于
 	\xx{   	\textcolor{red}{   中国长期以来一直是一个统一的大国 } }{   	\textcolor{red}{   中国人民顽强、持久的反抗}    }  {     	\textcolor{red}{ 帝国主义列强之间争夺中国的矛盾无法协调  } }  {  中西文化存在着巨大差异  } 
 	\note 漏选A了,记住她的脸,不忘记。  
 	
 	%第四题
 	\item 
 	19世纪70年代以后,王韬、薛福成、马建忠、郑观应等人不仅主张学习西方的科学技术,同时也要求吸纳西方的政治、经济学说。他们共同的特点是具有
 	\xx{\tr{比较强烈的反对外国侵略的爱国思想}}{\tr{希望中国独立富强的爱国思想}}{追求实行民主共和的制度的思想}{\tr{一定程度反对封建专职的民主思想}}
	\note 没选D,记下来,\tq{一定程度反对封建专职的民主思想}也对。
	
	%第五题
	\item 
	19世纪90年代末中国出现戊戌维新运动不是偶然的,它的发生有着深刻的历史背景,主要是
	\xx{中国民族资本主义已经成为中国经济的主要形式}{\tr{新兴的民族资产阶级迫切要求中国发展资本主义开辟道路}}{\tr{甲午战争惨败造成的民族危机激发来新的民族觉醒}}{\tr{资产阶级改良思想在中国迅速传播}}
	\note \tb{即便到了后期,中国民资也没有成为中国经济的主要形式}。
	
	%第六题
	\item 
	洋务运动期间,洋务派兴办的民用企业中,采用官督商办的企业基本是
	\xx{买办官僚性质的}{封建主义的}{半封建地半殖民地的}{\trq{资本主义性质的}}
	\note 这些官督商办的民用企业,虽然受官僚的控制,发展受控制,但基本上\tq{资本主义性质的近代企业}
	
	%第七题
	\item 
	甲午战争一役,洋务派经营多年的北洋海军全军覆没,标志着以``自强''`求富`'为目标的洋务运动的失败。决定洋务运动必然失败命运的原因是
	\xx{洋务运动对外国具有依赖性}{洋务派通过所掌握的国家权力集中力量优先发展军事工业 }{\trq{洋务运动的指导思想是``中学为体,西学为用''}}{洋务运动的目标是``自强,求富''}
	\note \tq{洋务运动的指导思想是``中学为体,西学为用'',即在封建主义思想的指导下,在维持的上层建筑、经济基础的条件下发展一些近代企业,为维护清朝的封建统治服务。}
	
	%第八题
	\item 
	康有为等维新派自身的局限性主要表现在
	\xx{\trq{不敢否定封建主义}}{\trq{对帝国主义抱有幻想}}{不敢鼓吹民权}{\trq{惧怕人民群众}}
	\note 记住她就行了。
	
	%第九题
	\item 
	戊戌维新运动是一次爱过救亡运动,维新派在民族危亡的关键时刻
	\xx{\trq{高举救亡图存的旗帜}}{\trq{要求通过变法,发展资本主义,使中国走上富强的道路}}{成功地建立起存在百日的君主立宪制度}{\trq{其政治实践和思想理论,贯穿着强烈的爱国主义精神,推动了中华民族的觉醒}}
	\note 戊戌变法未能成功地建立起资本主义的君主立宪制。
	
	%第10题
	\item 
	包括孙中山在内的许多中国革命先驱者早年也尝试采取和平的手段来推进中国的变革与进步,但最终还是走上了决心以革命的方法推翻清王朝的道路。促使孙中山改变思想,认为和平改革不能解决中国的问题的原因是
	\xx{\trq{孙中山关于改革的上书没有得到重视}}{\trq{孙中山发现清朝的腐败比他原先了解的要严重的多}}{孙中山得到了共产国际和中国共产党的真诚帮助}{资产阶级革命派在与资产阶级改良派的论战中得到了胜利}
	\note C、D与题意无关。
		
\end{enumerate}