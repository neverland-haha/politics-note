% !TEX encoding = UTF-8
%思修法基
\section{思修法基}
\textcolor{red}{\faBook 背就完事儿了}

\begin{enumerate}[align=hang, start=1]
	%第一题
	\item 法律是成文的道德,道德是内心的法律,法律和道德都具有规范社会行为、维护社会秩序的作用。以下既属于道德规范又是法律规范的是
	\xx{  	\textcolor{red}{  爱国主义 } }  {  	\textcolor{red}{  诚实守信 }   }   {  	\textcolor{red}{  尊重他人权利 }  }   {  	\textcolor{red}{  男女平等 }   } 
	\note 记就完事儿了。
	
	%第二题
	\item 人生目的是人生观的核心,这是因为
	\xx {  	\textcolor{red}{  人生目的决定人生道路 }}{ 	\textcolor{red}{  人生目的决定人生态度 }  }  { 	\textcolor{red}{   人生目的决定人生价值选择}   } { 人生目的决定人的本质 }  
	\note  \textcolor{ teal}{人的本质是社会关系的总和,不是由人生目的决定的}。
	
	%第三题
	\item 
	中国古人所说的``不义而富且贵,于我如浮云'',``一箪食,一瓢饮,在陋巷,人不堪其忧,回也不改其乐'',这两句话,表现的是中华民族崇尚精神的优秀传统中的
	\xx{对理想的不懈追求}{对道德修养和道德教化的重视}{\tr{对物质生活于精神相互关系的独到见解}}{对理想人格的推崇}
	\note \tq{题中所给的是对物质与精神生活相互关系的独到见解上}, \tb{``志士仁人,无求生以害仁,有杀生以成仁'';``兼相爱,交相利''}等等作为\tq{对理想的不懈追求上};而\tb{``立德、立功、立言''之首}体现了\tq{对道德修养和道德教化的重视上};\tb{``知之者不如好之者,好之者不如乐之者''、``可欲之谓善,有诸己之谓信''}表现为\tq{对理想人格的推崇}。
	
	%第四题
	\item 
	检验一个人对祖国忠诚程度的试金石是其
	\xx{\tq{对人民群众感情的深浅程度}}{对祖国大好河山的热爱程度}{对祖国灿烂文化的认同程度}{对民族优良文化的熟悉程度} 
	\note  \tb{爱国主义是}调节个人与祖国关系之间的\trq{道德要求、政治原则和法律规范},\tb{也是中华民族精神的核心}。
	\tq{对人民群众感情的深浅程度},\trq{是检验一个人对祖国忠诚程度的试金石},\trq{文化}是一个国家和民族得以延续的\trq{精神基因,是培养民族心理、民族个性、民族精神的``摇篮''}。\tq{爱自己的国家}是\trq{爱国主义的基本要求,爱祖国就是心系国家的前途和命运}。
	
	%第五题
	\item 
	近年来,一些地方政府出于社会利益考虑,准备在某些地点设立垃圾厂,核电厂、殡仪馆等设施,这往往引起当地居民或单位的不满,滋生``不要建在我家后院''的心理。``邻避效应''涉及个人与社会的关系问题。个人与社会的关系,最根本的是个人利益与社会利益的关系。社会利益是
	\xx{社会发展的根本目标}{\trq{作为社会成员的根本利益和长远利益的体现}}{\trq{个人利益得以实现的前提和基础}}{\trq{所有人利益的有机统一}}
	\note \tq{人的自我完善和全面发展、人生价值的实现,是社会发展的根本目标}。
	
	%第六题
	\item 
	比较客观、公正、准确地评价社会成员人生价值的大小,除了要掌握科学的标准外,还需要掌握恰当的评价方法。坚持
	\xx{\trq{能力大小与贡献尽力相统一}}{物质贡献与精神贡献相统一}{\trq{完善自身与贡献社会相统一}}{\trq{地位与影响相互统一}}
	\note 记就完事儿了!
	
	%第七题
	\item 
	马克思说:``历史承认那些为共同目标而因自己变得高尚的人是伟大人物;经验赞美那些为大多数人带来幸福的人是最幸福的人。''辩证对待人生矛盾,必须树立正确的幸福观。幸福
	\xx{  \trq{是一个总体范畴}}{都是尽善尽美的}{\trq{都是奋斗出来的,奋斗本身就是一种幸福}}{\trq{不能建立在损害社会利益整体利益和他人利益的基础智商}}
	\note \tq{幸福是相对的,不是尽善尽美的。}
	
	%第8题
	\item 
	2017年10月31日,党的十九大闭幕仅一周,中共中央总书记、国家主席、中共军委主席习近平带领十九届中共中央政治局专程从北京前往上海和浙江嘉兴,瞻仰上海中共一大会址和浙江嘉兴南湖红船。在瞻仰中共一大代表群浮雕像时,习近平对着雕像一一列数中共一大13名代表的姓名,感叹英雄辈出,也感叹大浪淘沙。个人坚守理想信念的强弱决定着人生命运的方向。理想信念的作用和意义在于
	\xx{\trq{昭示人生奋斗目标}}{\trq{提供人生前进动力}}{规划人生具体行程}{\trq{提高人生精神境界}} 
	\note 理想和信念的意义在于\tq{昭示奋斗目标、提供人生前进动力、提高精神境界}。
	
	\item 
	鲁迅曾说:``惟有民魂是值得宝贵的,惟有他发扬起来,中国才真有进步。''实现中国梦必须弘扬中国精神。中国精神是兴国强国之魂,是
	\xx{\trq{激发创新创造的精神动力}}{\trq{凝聚中国力量的精神纽带}}{\trq{}推进复兴伟业的精神定力}{政治文明建设的重要内容}
	\note 中国梦是\tq{激发创新创造的精神动力、凝聚中国力量的精神纽、推进复兴伟业的精神定力},D不符合题意。
	
	
\end{enumerate}
