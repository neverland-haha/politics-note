% !TEX encoding = UTF-8
\section{马哲部分}

\textcolor{red}{\faBook 马哲的部分最多也最难,需要理解的东西最多,也需要充分的理解才能做好题目。}

\begin{enumerate}[align=hang, start=1]
	%第一题
	\item 为马克思主义的产生提供了自然科学前提的有 
	\xx{相对论}{\textcolor{red}{细胞学说}}{\textcolor{red}{能量守恒与转化定律}}{\textcolor{red}{生物进化论}}%选择题调用直接使用本方式,会有自动缩进,也会有自动判断长度选择相应的形式
	\note 直接记住结论接可以了。
	
	
	%第二题
	\item 下列命题中,属于客观唯心主义哲学观点的有 
	\xx{物是感觉的复合}{世界是观念的集合}{\textcolor{red}{世界是绝对观念的外化}}{\textcolor{red}{世界是上帝意志的创造物}}%选择题调用直接使用本方式,会有自动缩进,也会有自动判断长度选择相应的形式
	\note B选项中的感觉是人的感觉,客观唯心把虚构出来的在人存在的客观精神当作世界的本原。
	
	
	%第三题
	\item 形而上学唯物主义物质观的缺陷在于
	\xx{\textcolor{red}{把某种特殊的物质形态误认为物质的一般}}{\textcolor{red}{割裂了自然界与人类社会的物质统一性}}{否认意识由物质决定}{\textcolor{red}{不了解人类对物质的认识是一个永无止境的发展过程}}%选择题调用直接使用本方式,会有自动缩进,也会有自动判断长度选择相应的形式
	\note 形而上学唯物主义把物质归结为原子,混淆了物质结构概念同哲学的物质范畴的区别,\textcolor{teal}{不了解人类对物质的认识是一个永无止境的发展过程。}把某种特殊的物质形态误认为物质的一般。\textcolor{teal}{割裂了自然界与人类社会的物质统一性}。
	
	
	%第四题
	\item 马克思主意的物质观认为,客观实在性是物质的唯一特性,既肯定了哲学物质范畴同自然科学物质结构理论的联系,又把它们区别开来,体现了唯物论和辩证法的统一,克服了形而上学唯物主义的缺陷。马克思主义物质观所体现的唯物辩证法是指
	\xx{\textcolor{red}{从个性中看到共性}}{\textcolor{red}{从相对中找到绝对}}{\textcolor{red}{从暂时中发现永恒}}{从局部中看到整体}
	\note 考研政治中,共性和个性、矛盾的普遍性和特殊性、一般和个别可以视为同一个层次的概念,而整体与部分是另一个层次的概念。
	
	
	%第五题
	\item 
	恩格斯于1820年11月28日出生在德国巴西门市的一个工厂主家庭。他称自己一生所做的事就是`` 拉第二小提琴''。恩格斯不仅与马克斯一起创立马克思主义,参加并指导国际工人运动,而且在传播和发展马克思主义方面做出了杰出的贡献。恩格斯全面阐述马克思主义理论体系的著作是 
	\xx{ 《共产党宣言》  }  {  《家庭、私有制和国家的起源》  }  {  \textcolor{red}{《反杜林论》}   }  { 《自然辩证法》   }
	\note  《德法年鉴》论文的发表->唯心主义向唯物主义、革命民主主义向共产主义转变,奠定了\textcolor{red}{思想前提} ,《德意志形态》->第一次比较系统的阐述了历史唯物主义的基本原理。《共产党宣言》->\textcolor{red}{马克思主义的公开问世}。 《资本论》-> 最厚重、最丰富的著作,被誉为\textcolor{red}{``工人阶级的圣经''}。《反杜林论》->全面阐述马克思主义理论体系,被称为\textcolor{red}{马克思主义的``百科全书''。}
	
	
	%第六题
	\item 
	2019年4月10日21时,全球多国科研人员合作的``世界视界望远镜''项目在全球六地同步举行发布会,发布了世界上首张黑洞图像,公布了人类首次拍到的黑洞照片。这是继2015年人类通过引力波探测``听到''两个黑洞合体之后,证明黑洞存在的直接``视界''证据。有科学家认为,这张看起来有点模糊的照片意义非凡,它再次验证了爱因斯坦广义相对论的语言是对的,并将进一步帮助科学家解答演化等一系列宇宙本质问题。人类首次``看到黑洞正面照表面'' 
	\xx{\textcolor{red}{空间的性质依赖于物质的分布及运动状态} }{\textcolor{red}{世界是物质的统一体}}{物质世界的客观存在与人的实践和认识水平有关}{\textcolor{red}{空间的观念随人的认识发展而拓展}}
	\note 没选D,\textcolor{teal}{物质运动和空间的客观实在是绝对的,物质运动时间和空间的具体特性是相对的。}
	%第七题
	\item 鲁迅说过:``描神画鬼,毫无对证,本可以专靠神思,所谓`天马行空'地挥写了。然而他们写出的却是三只眼、长颈子,也就是在正常的人体上增加了眼睛一只,拉长了颈子两三尺而已。''这段话说明,人们头脑中的鬼神观念
	\xx{ 是头脑中主观自生的  }{	\textcolor{red}{  是人脑对客观世界的歪曲反映 }   }{是人脑对鬼神的虚幻反映}{	\textcolor{red}{  可以从人世间找到它的原型 }} 
	\note  \textcolor{teal}{意识的主观性不仅表现为它对客观事物近似真实的反映,而且还可能表现为它是对客观对象的歪曲的或是虚幻的反映,但这种歪曲的或是虚幻的反映,也是对客观对象的反映,有其客观的``原型''}。歪曲反映而非虚幻反映,同时这种形象是人按照自己形象塑造出来的,而非头脑产生的,鬼神不是客观对象,所以不肯呢个是虚幻反映。
	\textcolor{blue}{意识,不管正确错误、先进落后,都是物质世界的主观映像。人的意识不管主观色彩多么浓厚,归根到底都有自己的客观``原型''}。
	
	
	\item 
	马克思说:``蜘蛛的活动和织工的活动相似,蜜蜂建筑蜂房的本领使人间的许多建筑师感到惭愧。但是最蹩脚的建筑师比最灵巧的秘方高明的地方,是他在用蜂蜡筑蜂房前,已经在自己的头脑中把它建成了。''这句话说明
	\xx   {\textcolor{red}{主观能动性是人区别于动物的特征}}{	\textcolor{red}{ 人的时间活动是有意识、有目的的 } }  
	{\textcolor{red}{人以自己的活动来改造世界,而动物只能以本能来适应环境 } }  {意识先于物质、决定物质}	
	\note 没选C项,\tq{与动物本能的、被动的适应性活动不同,人类的实践活动是一种有意识、有目的的活动,人以自己的活动来改造世界,而动物只能以本能来适应环境,主观能动性是人区别于动物的特征}。\tb{高级动物也有感觉和心理,但只有人才有意识。实践、意识都是人特有的,动物、机器人都没有}。\tb{同时人脑是意识的器官,但不是源泉,意识的源泉是客观世界}。
	
	
	%第9题
	\item 
	从物质与精神的关系来看,``画饼不能充饥''这是因为
	\xx{精神与物质不具有同一性}{精神对物质具有相对独立性}{\tr{观念的东西不能代替物质的东西}}{\tr{事物是人脑的反映不等同于事物自身}} 
	\note B项本身说法正确,但不符合题意,\tb{考研政治范围内,相对独立性是指,A决定B,但是B又具有自己特有的发展形式和规律,例如B的变化发展与A的变化发展不同步,B对A具有反作用等。}
	
	
	%第10题
	\item 
	马克思在《德意志意识形态》中写到:``因此,在这样的场合费尔巴哈从不谈人的世界,而是每次都求救于外部自然界,而且是那个尚未置于人的统治之下的自然界。但是,每当有了一项新的发明,每当工业向前前进一步,就有一块心的地盘从这个领域划出去,而能用来说明费尔巴哈这类论点的事例借以产生的基地,也越来越小了。''对这段话理解正确的有
	\xx{  \trq{自在自然日益转化为人化自然}}{\trq{实践是使物质世界分化为自然界与人类社会的历史前提和二者统一起来的现实基础}}{自然界不再具有客观实在性}{\trq{人类社会的存在和发展影响并不断改变着自然}} 
	\note 漏选了A,\tq{在实践活动过程中,物质世界被区分为自然界和人类社会两大领域。自然界是人生活于其中的客观世界,其中一部分是人类活动尚未触及的\tb{自在自然},另一部分是打上人类活动印记的\tb{人化自然}}。``每当工业前进一步,就有一块新的地盘转化为人化自然。A正确''。\tb{实践是物质世界分化为自然界与人类社会的\trq{历史前提},又是自然界与人类社会统一起来的\trq{现实基础}}。
	
	%第11题
	\item 
	陕西榆林地处于毛乌素沙漠边缘,当地凭借技术创新,把大片的荒沙地改成了高效农田。土地由坏变好,源于陕西省在沙漠治理上的技术创新。研究人员发现,毛乌素沙漠中含有大量的沙岩和沙子,这两种物质只要比例调配得当,就能改造好荒沙地/沙荒地改造的成功说明
	\xx{合理改造自然规律是人与自然和谐相处的关键}{科技创新是社会发展的根本动力}{\trq{认识世界是为了改造世界}}{\trq{人类可以通过劳动实践协调人与自然的关系}}
	\note \tq{规律是客观的,人既不能创造也不能改造自然},A中 \trq{改造} 两个字错了,\trq{社会基本矛盾}是社会发展的根本动力,\trq{阶级斗争、社会革命、改革、科技创新}是阶级社会发展的\trq{重要动力}。
	
\end{enumerate}
